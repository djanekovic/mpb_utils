\documentclass[utf8, seminar]{fer}
\usepackage{booktabs}
\usepackage{hyperref}
\usepackage{siunitx}

\begin{document}

\title{Analiza periodičkih struktura}
\author{Darko Janeković}
\voditelj{dr.sc. Dario Bojanjac}

\maketitle

\tableofcontents

\chapter{Uvod}
U ovom seminaru bit će riječi o analizi periodičkih struktura gdje će naglasak
biti na fotoničkim kristalima i primjeni numeričkih metoda na modeliranje
disperzijskog dijagrama. Programska biblioteka koja se koristi za numeričko
modeliranje je \href{https://github.com/stevengj/mpb}{MPB}. Prije svega, bit će
iznesena matematička podloga potrebna za efikasno modeliranje širenja vala u
periodičkoj strukturi. U poglavljima nakon uvoda bit će iznesena primjena
fotoničkih kristala, te će biti pobrojane numeričke metode korištene u svrhu
modeliranja problema koji opisuju periodičke strukture.

\chapter{Matematičko modeliranje širenja vala u periodičkoj strukturi}

Periodička struktura je pravilna struktura u kojoj se elementi na neki način
periodički ponavljaju. U nastavku će biti razmatrani fotonički kristali koji se
mogu razmati kao kristalne rešetke. Fotonički kristali su periodični na način
da zadovavaju diskretnu translacijsku simetriju. Diskretna translacijska
simetrija definirana je kao ${f(\mathbf{r}) = f(\mathbf{r} \pm \mathbf{a})}$
odnosno ${f(\mathbf{r}) = f(\mathbf{r} + \mathbf{R})}$ gdje je ${\mathbf{R} =
n\mathbf{a}}$, a $n \in \mathbb{Z}$.

Vektor $\mathbf{a}$ je primitivni vektor rešetke dok se njegova duljina naziva
konstanta rešetke. Ćelija za koju tvrdimo da se iznova ponavlja i tvori
kristalnu rešetku naziva se jedinična ćelija.
%todo: nastavi možda detaljnije o rešetci i nešto o simetriji općenito.
%Nastavak blockov teorem i onda spoji sve s valnom i izvodom valne za
%nehomogeni dielektrik



\chapter{Zaključak}
Zaključak.

\bibliography{literatura}
\bibliographystyle{fer}

\chapter{Sažetak}
Sažetak.

\end{document}
